% Options for packages loaded elsewhere
\PassOptionsToPackage{unicode}{hyperref}
\PassOptionsToPackage{hyphens}{url}
\PassOptionsToPackage{dvipsnames,svgnames,x11names}{xcolor}
%
\documentclass[
  letterpaper,
  DIV=11,
  numbers=noendperiod]{scrreprt}

\usepackage{amsmath,amssymb}
\usepackage[]{libertinus}
\usepackage{iftex}
\ifPDFTeX
  \usepackage[T1]{fontenc}
  \usepackage[utf8]{inputenc}
  \usepackage{textcomp} % provide euro and other symbols
\else % if luatex or xetex
  \usepackage{unicode-math}
  \defaultfontfeatures{Scale=MatchLowercase}
  \defaultfontfeatures[\rmfamily]{Ligatures=TeX,Scale=1}
\fi
% Use upquote if available, for straight quotes in verbatim environments
\IfFileExists{upquote.sty}{\usepackage{upquote}}{}
\IfFileExists{microtype.sty}{% use microtype if available
  \usepackage[]{microtype}
  \UseMicrotypeSet[protrusion]{basicmath} % disable protrusion for tt fonts
}{}
\makeatletter
\@ifundefined{KOMAClassName}{% if non-KOMA class
  \IfFileExists{parskip.sty}{%
    \usepackage{parskip}
  }{% else
    \setlength{\parindent}{0pt}
    \setlength{\parskip}{6pt plus 2pt minus 1pt}}
}{% if KOMA class
  \KOMAoptions{parskip=half}}
\makeatother
\usepackage{xcolor}
\setlength{\emergencystretch}{3em} % prevent overfull lines
\setcounter{secnumdepth}{5}
% Make \paragraph and \subparagraph free-standing
\ifx\paragraph\undefined\else
  \let\oldparagraph\paragraph
  \renewcommand{\paragraph}[1]{\oldparagraph{#1}\mbox{}}
\fi
\ifx\subparagraph\undefined\else
  \let\oldsubparagraph\subparagraph
  \renewcommand{\subparagraph}[1]{\oldsubparagraph{#1}\mbox{}}
\fi

\usepackage{color}
\usepackage{fancyvrb}
\newcommand{\VerbBar}{|}
\newcommand{\VERB}{\Verb[commandchars=\\\{\}]}
\DefineVerbatimEnvironment{Highlighting}{Verbatim}{commandchars=\\\{\}}
% Add ',fontsize=\small' for more characters per line
\newenvironment{Shaded}{}{}
\newcommand{\AlertTok}[1]{\textcolor[rgb]{1.00,0.33,0.33}{\textbf{#1}}}
\newcommand{\AnnotationTok}[1]{\textcolor[rgb]{0.42,0.45,0.49}{#1}}
\newcommand{\AttributeTok}[1]{\textcolor[rgb]{0.84,0.23,0.29}{#1}}
\newcommand{\BaseNTok}[1]{\textcolor[rgb]{0.00,0.36,0.77}{#1}}
\newcommand{\BuiltInTok}[1]{\textcolor[rgb]{0.84,0.23,0.29}{#1}}
\newcommand{\CharTok}[1]{\textcolor[rgb]{0.01,0.18,0.38}{#1}}
\newcommand{\CommentTok}[1]{\textcolor[rgb]{0.42,0.45,0.49}{#1}}
\newcommand{\CommentVarTok}[1]{\textcolor[rgb]{0.42,0.45,0.49}{#1}}
\newcommand{\ConstantTok}[1]{\textcolor[rgb]{0.00,0.36,0.77}{#1}}
\newcommand{\ControlFlowTok}[1]{\textcolor[rgb]{0.84,0.23,0.29}{#1}}
\newcommand{\DataTypeTok}[1]{\textcolor[rgb]{0.84,0.23,0.29}{#1}}
\newcommand{\DecValTok}[1]{\textcolor[rgb]{0.00,0.36,0.77}{#1}}
\newcommand{\DocumentationTok}[1]{\textcolor[rgb]{0.42,0.45,0.49}{#1}}
\newcommand{\ErrorTok}[1]{\textcolor[rgb]{1.00,0.33,0.33}{\underline{#1}}}
\newcommand{\ExtensionTok}[1]{\textcolor[rgb]{0.84,0.23,0.29}{\textbf{#1}}}
\newcommand{\FloatTok}[1]{\textcolor[rgb]{0.00,0.36,0.77}{#1}}
\newcommand{\FunctionTok}[1]{\textcolor[rgb]{0.44,0.26,0.76}{#1}}
\newcommand{\ImportTok}[1]{\textcolor[rgb]{0.01,0.18,0.38}{#1}}
\newcommand{\InformationTok}[1]{\textcolor[rgb]{0.42,0.45,0.49}{#1}}
\newcommand{\KeywordTok}[1]{\textcolor[rgb]{0.84,0.23,0.29}{#1}}
\newcommand{\NormalTok}[1]{\textcolor[rgb]{0.14,0.16,0.18}{#1}}
\newcommand{\OperatorTok}[1]{\textcolor[rgb]{0.14,0.16,0.18}{#1}}
\newcommand{\OtherTok}[1]{\textcolor[rgb]{0.44,0.26,0.76}{#1}}
\newcommand{\PreprocessorTok}[1]{\textcolor[rgb]{0.84,0.23,0.29}{#1}}
\newcommand{\RegionMarkerTok}[1]{\textcolor[rgb]{0.42,0.45,0.49}{#1}}
\newcommand{\SpecialCharTok}[1]{\textcolor[rgb]{0.00,0.36,0.77}{#1}}
\newcommand{\SpecialStringTok}[1]{\textcolor[rgb]{0.01,0.18,0.38}{#1}}
\newcommand{\StringTok}[1]{\textcolor[rgb]{0.01,0.18,0.38}{#1}}
\newcommand{\VariableTok}[1]{\textcolor[rgb]{0.89,0.38,0.04}{#1}}
\newcommand{\VerbatimStringTok}[1]{\textcolor[rgb]{0.01,0.18,0.38}{#1}}
\newcommand{\WarningTok}[1]{\textcolor[rgb]{1.00,0.33,0.33}{#1}}

\providecommand{\tightlist}{%
  \setlength{\itemsep}{0pt}\setlength{\parskip}{0pt}}\usepackage{longtable,booktabs,array}
\usepackage{calc} % for calculating minipage widths
% Correct order of tables after \paragraph or \subparagraph
\usepackage{etoolbox}
\makeatletter
\patchcmd\longtable{\par}{\if@noskipsec\mbox{}\fi\par}{}{}
\makeatother
% Allow footnotes in longtable head/foot
\IfFileExists{footnotehyper.sty}{\usepackage{footnotehyper}}{\usepackage{footnote}}
\makesavenoteenv{longtable}
\usepackage{graphicx}
\makeatletter
\def\maxwidth{\ifdim\Gin@nat@width>\linewidth\linewidth\else\Gin@nat@width\fi}
\def\maxheight{\ifdim\Gin@nat@height>\textheight\textheight\else\Gin@nat@height\fi}
\makeatother
% Scale images if necessary, so that they will not overflow the page
% margins by default, and it is still possible to overwrite the defaults
% using explicit options in \includegraphics[width, height, ...]{}
\setkeys{Gin}{width=\maxwidth,height=\maxheight,keepaspectratio}
% Set default figure placement to htbp
\makeatletter
\def\fps@figure{htbp}
\makeatother
\newlength{\cslhangindent}
\setlength{\cslhangindent}{1.5em}
\newlength{\csllabelwidth}
\setlength{\csllabelwidth}{3em}
\newlength{\cslentryspacingunit} % times entry-spacing
\setlength{\cslentryspacingunit}{\parskip}
\newenvironment{CSLReferences}[2] % #1 hanging-ident, #2 entry spacing
 {% don't indent paragraphs
  \setlength{\parindent}{0pt}
  % turn on hanging indent if param 1 is 1
  \ifodd #1
  \let\oldpar\par
  \def\par{\hangindent=\cslhangindent\oldpar}
  \fi
  % set entry spacing
  \setlength{\parskip}{#2\cslentryspacingunit}
 }%
 {}
\usepackage{calc}
\newcommand{\CSLBlock}[1]{#1\hfill\break}
\newcommand{\CSLLeftMargin}[1]{\parbox[t]{\csllabelwidth}{#1}}
\newcommand{\CSLRightInline}[1]{\parbox[t]{\linewidth - \csllabelwidth}{#1}\break}
\newcommand{\CSLIndent}[1]{\hspace{\cslhangindent}#1}

% Preámbulo
\usepackage{comment}
\usepackage{marvosym}
\usepackage{float} % Para controlar la posición de las tablas y figuras
\usepackage{makeidx}

\usepackage{mathptmx}
\usepackage{amsmath} % Para utilizar símbolos matemáticos y ecuaciones
\usepackage{setspace}
\usepackage{lipsum} % Crear texto RAMDOM
\usepackage{multirow} % Agregar TABLAS 
\usepackage{array} % Dar formato a las TABLAS


\usepackage{booktabs} % Para crear tablas bonitas
\usepackage{caption} % Para personalizar las leyendas de las tablas y figuras
\usepackage{hyperref} % Para crear enlaces dentro del documento

%Para figuras
\usepackage{graphicx} % Para incluir imágenes
\usepackage{subcaption} % Insertar SubImagenes
\usepackage{tikz} % Para generar gráficos
\usepackage{pgfplots} % Para generar gráficos en LaTeX

\KOMAoption{captions}{tableheading}
\makeatletter
\makeatother
\makeatletter
\@ifpackageloaded{bookmark}{}{\usepackage{bookmark}}
\makeatother
\makeatletter
\@ifpackageloaded{caption}{}{\usepackage{caption}}
\AtBeginDocument{%
\ifdefined\contentsname
  \renewcommand*\contentsname{Tabla de contenidos}
\else
  \newcommand\contentsname{Tabla de contenidos}
\fi
\ifdefined\listfigurename
  \renewcommand*\listfigurename{Listado de Figuras}
\else
  \newcommand\listfigurename{Listado de Figuras}
\fi
\ifdefined\listtablename
  \renewcommand*\listtablename{Listado de Tablas}
\else
  \newcommand\listtablename{Listado de Tablas}
\fi
\ifdefined\figurename
  \renewcommand*\figurename{Figura}
\else
  \newcommand\figurename{Figura}
\fi
\ifdefined\tablename
  \renewcommand*\tablename{Tabla}
\else
  \newcommand\tablename{Tabla}
\fi
}
\@ifpackageloaded{float}{}{\usepackage{float}}
\floatstyle{ruled}
\@ifundefined{c@chapter}{\newfloat{codelisting}{h}{lop}}{\newfloat{codelisting}{h}{lop}[chapter]}
\floatname{codelisting}{Listado}
\newcommand*\listoflistings{\listof{codelisting}{Listado de Listados}}
\usepackage{amsthm}
\theoremstyle{definition}
\newtheorem{example}{Ejemplo}[chapter]
\theoremstyle{definition}
\newtheorem{definition}{Definición}[chapter]
\theoremstyle{remark}
\renewcommand*{\proofname}{Prueba}
\newtheorem*{remark}{Observación}
\newtheorem*{solution}{Solución}
\makeatother
\makeatletter
\@ifpackageloaded{caption}{}{\usepackage{caption}}
\@ifpackageloaded{subcaption}{}{\usepackage{subcaption}}
\makeatother
\makeatletter
\@ifpackageloaded{tcolorbox}{}{\usepackage[many]{tcolorbox}}
\makeatother
\makeatletter
\@ifundefined{shadecolor}{\definecolor{shadecolor}{rgb}{.97, .97, .97}}
\makeatother
\makeatletter
\makeatother
\ifLuaTeX
\usepackage[bidi=basic]{babel}
\else
\usepackage[bidi=default]{babel}
\fi
\babelprovide[main,import]{spanish}
% get rid of language-specific shorthands (see #6817):
\let\LanguageShortHands\languageshorthands
\def\languageshorthands#1{}
\ifLuaTeX
  \usepackage{selnolig}  % disable illegal ligatures
\fi
\usepackage[]{biblatex}
\addbibresource{references.bib}
\IfFileExists{bookmark.sty}{\usepackage{bookmark}}{\usepackage{hyperref}}
\IfFileExists{xurl.sty}{\usepackage{xurl}}{} % add URL line breaks if available
\urlstyle{same} % disable monospaced font for URLs
% Make links footnotes instead of hotlinks:
\DeclareRobustCommand{\href}[2]{#2\footnote{\url{#1}}}
\hypersetup{
  pdftitle={Investigacion operativa},
  pdfauthor={Elmer Edison Achalma Mendoza},
  pdflang={es},
  colorlinks=true,
  linkcolor={blue},
  filecolor={Maroon},
  citecolor={Blue},
  urlcolor={Blue},
  pdfcreator={LaTeX via pandoc}}

\title{Investigacion operativa}
\author{Elmer Edison Achalma Mendoza}
\date{1/8/23}

\begin{document}
\maketitle
\ifdefined\Shaded\renewenvironment{Shaded}{\begin{tcolorbox}[interior hidden, frame hidden, sharp corners, enhanced, borderline west={3pt}{0pt}{shadecolor}, breakable, boxrule=0pt]}{\end{tcolorbox}}\fi

\renewcommand*\contentsname{Tabla de contenidos}
{
\hypersetup{linkcolor=}
\setcounter{tocdepth}{2}
\tableofcontents
}
\bookmarksetup{startatroot}

\hypertarget{preface}{%
\chapter*{Preface}\label{preface}}
\addcontentsline{toc}{chapter}{Preface}

\markboth{Preface}{Preface}

This is a Quarto book.

To learn more about Quarto books visit
\url{https://quarto.org/docs/books}.

\begin{Shaded}
\begin{Highlighting}[]
\DecValTok{1} \SpecialCharTok{+} \DecValTok{1}
\end{Highlighting}
\end{Shaded}

\begin{verbatim}
[1] 2
\end{verbatim}

\bookmarksetup{startatroot}

\hypertarget{introduction}{%
\chapter*{Introduction}\label{introduction}}
\addcontentsline{toc}{chapter}{Introduction}

\markboth{Introduction}{Introduction}

This is a book created from markdown and executable code.

See \textcite{knuth84} for additional discussion of literate
programming.

La investigación operativa es una disciplina que se enfoca en el
análisis y resolución de problemas de optimización en diferentes
ámbitos, tales como la empresa, la industria y el gobierno. Esta
disciplina se basa en el uso de métodos matemáticos y estadísticos para
encontrar la solución óptima a un problema específico, teniendo en
cuenta diferentes factores y restricciones.

Para aprender investigación operativa, es importante tener conocimientos
básicos de matemáticas, especialmente en áreas como álgebra y cálculo.
También es útil conocer algoritmos y estructuras de datos, así como
tener habilidades de programación para poder implementar y probar
soluciones.

El proceso de aprendizaje de investigación operativa suele incluir la
exposición a diferentes tipos de problemas y el análisis de cómo se
pueden resolver de manera óptima. También se pueden estudiar métodos y
técnicas específicas, como programación lineal, programación no lineal,
teoría de colas y teoría de juegos.

Además de los conocimientos teóricos, es importante desarrollar
habilidades prácticas en la resolución de problemas, lo que incluye la
identificación de las variables y restricciones relevantes, la
formulación de un modelo matemático y la implementación de una solución.
La investigación operativa también implica la toma de decisiones y la
evaluación de diferentes escenarios y alternativas, por lo que es
importante desarrollar habilidades de análisis y pensamiento crítico.

\bookmarksetup{startatroot}

\hypertarget{summary}{%
\chapter*{Summary}\label{summary}}
\addcontentsline{toc}{chapter}{Summary}

\markboth{Summary}{Summary}

In summary, this book has no content whatsoever.

\begin{Shaded}
\begin{Highlighting}[]
\DecValTok{1} \SpecialCharTok{+} \DecValTok{1}
\end{Highlighting}
\end{Shaded}

\begin{verbatim}
[1] 2
\end{verbatim}

\part{Capítulos}

\hypertarget{introducciuxf3n-a-la-investigaciuxf3n-operativa}{%
\chapter{Introducción a la investigación
operativa}\label{introducciuxf3n-a-la-investigaciuxf3n-operativa}}

\#io \#book

\leavevmode\vadjust pre{\hypertarget{def-io}{}}%
\begin{definition}[Investigacion de operaciones]\label{def-io}

La Investigación de Operaciones es una disciplina que se ocupa del
estudio y análisis de sistemas de producción y servicios. Su objetivo es
encontrar las mejores soluciones posibles a los problemas que surgen en
la toma de decisiones en dichos sistemas. La disciplina se originó
durante la Segunda Guerra Mundial, cuando se necesitaba optimizar la
producción de material bélico y mejorar la eficiencia logística. Desde
entonces, ha tenido un papel importante en la mejora de la productividad
y la eficiencia en diversos campos, como la manufactura, la ingeniería,
la logística, la finanzas, la salud y la educación, entre otros:

\end{definition}

La Investigación de Operaciones (IO) es un campo de la matemática y la
informática que se dedica a la resolución de problemas prácticos
mediante el uso de modelos matemáticos y algoritmos. Estos problemas
pueden ser de diversa índole, tales como la optimización de recursos en
una empresa, el diseño de un plan de producción eficiente, la gestión de
inventarios, entre otros. La IO utiliza técnicas formales para analizar
y resolver estos problemas de manera óptima y eficiente.

\hypertarget{tuxe9cnicas-y-aplicaciones-de-la-io}{%
\section{Técnicas y aplicaciones de la
IO}\label{tuxe9cnicas-y-aplicaciones-de-la-io}}

La Investigación de operaciones (IO) es una disciplina que se ocupa del
estudio y análisis de sistemas complejos y procesos de decisión en
diversos contextos, con el objetivo de mejorar su eficiencia y
efectividad. Para ello, se utilizan diversas técnicas y herramientas
matemáticas y estadísticas. Algunas de las técnicas más comunes en la IO
son:

\begin{itemize}
\tightlist
\item
  Programación lineal: permite optimizar una función objetivo, sujeta a
  un conjunto de restricciones, mediante el uso de técnicas de
  optimización matemática.
\item
  Análisis de transporte: se utiliza para determinar la forma más
  eficiente de mover un producto o material desde un origen hasta un
  destino, minimizando costos de transporte.
\item
  Programación dinámica: permite modelar y resolver problemas que
  involucran decisiones a lo largo del tiempo, considerando factores
  como el costo y la incertidumbre.
\item
  Simulación: es una técnica que permite evaluar el comportamiento de un
  sistema o proceso a través de la repetición de escenarios hipotéticos.
\item
  Teoría de colas: se utiliza para modelar y analizar sistemas de
  atención y espera, como por ejemplo, una fila de personas esperando
  ser atendidas en una institución bancaria.
\end{itemize}

Algunas de las aplicaciones de la IO son:

\begin{itemize}
\tightlist
\item
  Diseño y gestión de sistemas de producción y distribución.
\item
  Diseño y optimización de sistemas de transporte y logística.
\item
  Gestión de inventarios y suministros.
\item
  Diseño y gestión de sistemas de atención y servicio al cliente.
\item
  Gestión de proyectos y programas.
\item
  Diseño y optimización de sistemas de información y tecnología.
\end{itemize}

\hypertarget{fundamentos-de-uxe1lgebra-matricial}{%
\section{Fundamentos de álgebra
matricial}\label{fundamentos-de-uxe1lgebra-matricial}}

Los fundamentos de álgebra matricial son un conjunto de conceptos y
herramientas matemáticas que se utilizan para trabajar con matrices, es
decir, arreglos de números dispuestos en filas y columnas. Algunos de
los conceptos fundamentales de álgebra matricial son:

\begin{itemize}
\item
  Suma y resta de matrices: para sumar o restar dos matrices, deben
  tener la misma dimensión (es decir, el mismo número de filas y
  columnas). Se realiza elemento por elemento, es decir, se suman o
  restan los elementos correspondientes de cada matriz.
\item
  Multiplicación de una matriz por un escalar: para multiplicar una
  matriz por un escalar (un número), se multiplican todos los elementos
  de la matriz por ese escalar.
\item
  Multiplicación de matrices: para multiplicar dos matrices, deben
  cumplirse ciertas condiciones. La matriz resultante tendrá tantas
  filas como la primera matriz y tantas columnas como la segunda. El
  elemento ij-ésimo de la matriz resultante se obtiene multiplicando
  cada elemento de la i-ésima fila de la primera matriz por el elemento
  correspondiente de la j-ésima columna de la segunda matriz, y sumando
  todos esos productos.
\item
  Transpuesta de una matriz: la transpuesta de una matriz A es una
  matriz A\^{}T que se obtiene intercambiando filas por columnas en A.
\item
  Matriz inversa: una matriz inversa es una matriz A\^{}(-1) que, al
  multiplicarla por la matriz A original, da como resultado la matriz
  identidad.
\item
  Determinante de una matriz: el determinante de una matriz es un número
  que se puede calcular a partir de los elementos de la matriz y que
  tiene ciertas propiedades. Por ejemplo, si el determinante de una
  matriz es cero, la matriz no tiene inversa.
\end{itemize}

\hypertarget{matriz-inversa-seguxfan-el-muxe9todo-de-gauss}{%
\section{Matriz inversa según el método de
Gauss}\label{matriz-inversa-seguxfan-el-muxe9todo-de-gauss}}

Para obtener la matriz inversa de una matriz A mediante el método de
Gauss, se puede seguir el siguiente proceso:

\begin{enumerate}
\def\labelenumi{\arabic{enumi}.}
\item
  Escribir la matriz A junto con la matriz identidad (I) de tamaño igual
  a la matriz A. Esto se conoce como la matriz aumentada (A\textbar I).
\item
  Aplicar las operaciones elementales necesarias para convertir la
  matriz A en la matriz diagonal (en la que sólo hay elementos distintos
  de cero en la diagonal principal). Estas operaciones deben aplicarse
  también a la matriz identidad para mantener la relación entre ambas.
\item
  Una vez que se ha obtenido la matriz diagonal, se puede obtener la
  matriz inversa aplicando las mismas operaciones elementales a la
  matriz identidad, pero en orden inverso y con los coeficientes
  cambiados de signo.
\end{enumerate}

\begin{Shaded}
\begin{Highlighting}[]
\CommentTok{\# Para calcular la matriz inversa de una matriz A en R, puedes utilizar la }
\CommentTok{\# función solve() de la siguiente manera:}
\CommentTok{\# Primero, debes cargar la matriz A}

\NormalTok{A }\OtherTok{\textless{}{-}} \FunctionTok{matrix}\NormalTok{(}\FunctionTok{c}\NormalTok{(}\DecValTok{1}\NormalTok{, }\DecValTok{2}\NormalTok{, }\DecValTok{3}\NormalTok{, }\DecValTok{4}\NormalTok{), }\AttributeTok{nrow =} \DecValTok{2}\NormalTok{, }\AttributeTok{ncol =} \DecValTok{2}\NormalTok{)}
\CommentTok{\# A}
\CommentTok{\# Luego, puedes calcular la inversa de A utilizando solve()}

\NormalTok{A\_inv }\OtherTok{\textless{}{-}} \FunctionTok{solve}\NormalTok{(A)}
\CommentTok{\# A\_inv}

\CommentTok{\# También puedes utilizar la función ginv() del paquete MASS, que utiliza el }
\CommentTok{\# método de Moore{-}Penrose para calcular la inversa generalizada de una matriz:}

\CommentTok{\# Primero, debes cargar la matriz A y el paquete MASS}

\NormalTok{A }\OtherTok{\textless{}{-}} \FunctionTok{matrix}\NormalTok{(}\FunctionTok{c}\NormalTok{(}\DecValTok{1}\NormalTok{, }\DecValTok{2}\NormalTok{, }\DecValTok{3}\NormalTok{, }\DecValTok{4}\NormalTok{), }\AttributeTok{nrow =} \DecValTok{2}\NormalTok{, }\AttributeTok{ncol =} \DecValTok{2}\NormalTok{)}
\FunctionTok{library}\NormalTok{(MASS)}

\CommentTok{\# Luego, puedes calcular la inversa de A utilizando ginv()}
\NormalTok{A\_inv }\OtherTok{\textless{}{-}} \FunctionTok{ginv}\NormalTok{(A)}
\end{Highlighting}
\end{Shaded}

\leavevmode\vadjust pre{\hypertarget{exm-ejemplo1}{}}%
\begin{example}[]\label{exm-ejemplo1}

Para obtener la matriz inversa de A.

\begin{equation*}
A_{2 \times 2}=
\begin{pmatrix}
1 & 2 \\
3 & 4 
\end{pmatrix} _{2 \times 2}
\end{equation*}

Se puede seguir el siguiente proceso:

La matriz aumentada es A\textbar I =

\begin{equation*}
A_{2 \times 2}=
\begin{pmatrix}
1 & 2 & 1 & 0 \\
3 & 4 & 0 & 1 
\end{pmatrix} _{2 \times 2}
\end{equation*}

Aplicar una operación elemental de intercambio de filas para convertir
el primer elemento de la matriz A en 1:

\begin{equation*}
A_{2 \times 2}=
\begin{pmatrix}
3 & 4 & 0 & 1 \\
1 & 2 & 1 & 0 
\end{pmatrix} _{2 \times 2}
\end{equation*}

Aplicar una operación elemental de multiplicación de fila por un escalar
para que el primer elemento de la segunda fila sea -1:

\begin{equation*}
A_{2 \times 2}=
\begin{pmatrix}
3 & 4 & 0 & 1 \\
-1 & -2 & -1 & 0 
\end{pmatrix} _{2 \times 2}
\end{equation*}

Aplicar una operación elemental de suma de filas para eliminar el primer
elemento de la segunda fila:

\begin{equation*}
A_{2 \times 2}=
\begin{pmatrix}
3 & 4 & 0 & 1 \\
0 & 0 & 0 & 1 
\end{pmatrix} _{2 \times 2}
\end{equation*}

Aplicar las operaciones elementales necesarias a la matriz identidad
para obtener la matriz inversa:

\begin{equation*}
A_{2 \times 2}=
\begin{pmatrix}
1/3 & -2/3 & 0 & -1/3 \\
0 & 0 & 0 & 1 
\end{pmatrix} _{2 \times 2}
\end{equation*}

Por tanto, la matriz inversa de A es:

\begin{equation*}
A_{2 \times 2}=
\begin{pmatrix}
1/3 & -2/3 \\
0 & 1 
\end{pmatrix} _{2 \times 2}
\end{equation*}

\end{example}

\hypertarget{optmizaciuxf3n-lineal-contuxednua}{%
\chapter{Optmización lineal
contínua}\label{optmizaciuxf3n-lineal-contuxednua}}

La optimización lineal contínua es una técnica matemática que se utiliza
para encontrar el valor óptimo de una función lineal sujeta a ciertas
restricciones o condiciones. Esta técnica se aplica a problemas que
involucran variables que pueden tomar cualquier valor en un rango
específico, en lugar de solo valores discretos. Los problemas de
optimización lineal contínua se pueden resolver utilizando diferentes
métodos, como el método simplex o el algoritmo de punto interior. La
optimización lineal contínua se utiliza en diversas áreas, como la
ingeniería, la economía y la ciencia de la computación.

\hypertarget{modelo-general-de-programaciuxf3n-lineal-pl}{%
\section{Modelo general de Programación Lineal
(PL)}\label{modelo-general-de-programaciuxf3n-lineal-pl}}

La Programación Lineal (PL) es una técnica de optimización que se
utiliza para encontrar el valor óptimo de una función de varias
variables (llamada función objetivo) sujeta a un conjunto de
restricciones. En el modelo general de PL, la función objetivo y las
restricciones están expresadas mediante ecuaciones o inecuaciones que
involucran una combinación lineal de las variables. La solución del
problema de PL consiste en encontrar los valores de las variables que
optimizan la función objetivo, sujeto a cumplir con las restricciones.

El modelo general de Programación Lineal (PL) puede ser representado de
la siguiente manera en LaTeX:

\hypertarget{variables-de-decisiuxf3n}{%
\section{Variables de decisión}\label{variables-de-decisiuxf3n}}

\hypertarget{funciuxf3n-objetivo}{%
\section{Función Objetivo}\label{funciuxf3n-objetivo}}

\hypertarget{restricciones}{%
\section{Restricciones}\label{restricciones}}

\hypertarget{formulaciuxf3n-de-modelos-de-pl}{%
\section{Formulación de modelos de
PL}\label{formulaciuxf3n-de-modelos-de-pl}}

\hypertarget{algunos-casos-de-estudios-cluxe1sicos}{%
\section{Algunos casos de estudios
clásicos}\label{algunos-casos-de-estudios-cluxe1sicos}}

\hypertarget{muxe9todo-gruxe1fico}{%
\section{Método Gráfico}\label{muxe9todo-gruxe1fico}}

problema en forma estándar

Representar gráficamente las restricciones

\begin{tikzpicture}
\begin{axis}[
    axis lines=middle,
    xmin=0, xmax=80,
    ymin=0, ymax=140,
    xtick={0,40,80},
    ytick={0,80,120,140},
    xlabel={$x_1$},
    ylabel={$x_2$}
]
\addplot[domain=0:80, samples=2, thick, blue]{80-x};
\addplot[domain=0:80, samples=2, thick, red]{(220-3*x)/2};
\addplot[domain=0:80, samples=2, thick, green]{(210-2*x)/3};
\end{axis}
\end{tikzpicture}

Identificar el conjunto solución factible.

\begin{tikzpicture}
\begin{axis}[
    axis lines=middle,
    xmin=0, xmax=80,
    ymin=0, ymax=140,
    xtick={0,40,80},
    ytick={0,80,120,140},
    xlabel={$x_1$},
    ylabel={$x_2$},
    fill=gray, opacity=0.3
]
\addplot[domain=0:40, samples=2, thick, blue]{80-x};
\addplot[domain=40:80, samples=2, thick, blue]{120-3*x/2};
\addplot[domain=0:60, samples=2, thick, red]{(220-3*x)/2};
\addplot[domain=60:80, samples=2, thick, red]{(210-2*x)/3};
\addplot[domain=0:80, samples=2, thick, green]{(210-2*x)/3};
\filldraw[green] (0,80) -- (40,120) -- (60,80) -- cycle;
\end{axis}
\end{tikzpicture}

Identificar los puntos extremos del conjunto solución factible.

\begin{tikzpicture}
\begin{axis}[    axis lines=middle,    xmin=0, xmax=80,    ymin=0, ymax=140,    xtick={0,40,80},    ytick={0,80,120,140},    xlabel={$x_1$},    ylabel={$x_2$},    fill=gray, opacity=0.3]
\addplot[domain=0:40, samples=2, thick, blue]{80-x};
\addplot[domain=40:80, samples=2, thick, blue]{120-3*x/2};
\addplot[domain=0:60, samples=2, thick, red]{(220-3*x)/2};
\addplot[domain=60:80, samples=2, thick, red]{(210-2*x)/3};
\addplot[domain=0:80, samples=2, thick, green]{(210-2*x)/3};
\filldraw[green] (0,80) -- (40,120) -- (60,80) -- cycle;
\draw[black,dashed] (0,80) -- (40,120) -- (60,80);
\draw[black,dashed] (0,80) -- (60,80);
\draw[black,dashed] (40,120) -- (60,80);
\filldraw[black] (0,80) circle (2pt);
\filldraw[black] (40,120) circle (2pt);
\filldraw[black] (60,80) circle (2pt);
\end{axis}
\end{tikzpicture}

\hypertarget{casos-especiales-del-muxe9todo-gruxe1fico}{%
\subsection{Casos Especiales del Método
Gráfico}\label{casos-especiales-del-muxe9todo-gruxe1fico}}

\hypertarget{ejercicios-de-aplicaciuxf3n}{%
\section{Ejercicios de aplicación}\label{ejercicios-de-aplicaciuxf3n}}

\hypertarget{problemas-de-transporte-y-de-asignaciuxf3n}{%
\chapter{Problemas de transporte y de
asignación}\label{problemas-de-transporte-y-de-asignaciuxf3n}}

\hypertarget{muxe9todo-simplex}{%
\section{Método Simplex}\label{muxe9todo-simplex}}

\hypertarget{procedimiento-del-muxe9todo-simplex}{%
\subsection{Procedimiento del Método
simplex}\label{procedimiento-del-muxe9todo-simplex}}

\hypertarget{pasos-para-aplicar-muxe9todo-simplex}{%
\subsection{pasos para aplicar método
simplex}\label{pasos-para-aplicar-muxe9todo-simplex}}

\hypertarget{tablero-simplex}{%
\subsection{tablero Simplex}\label{tablero-simplex}}

\hypertarget{formas-de-presentaciuxf3n-de-un-programa-lineal}{%
\subsection{formas de presentación de un programa
lineal}\label{formas-de-presentaciuxf3n-de-un-programa-lineal}}

\hypertarget{forma-canuxf3nica-y-estuxe1ndar}{%
\subsection{forma canónica y
estándar}\label{forma-canuxf3nica-y-estuxe1ndar}}

\hypertarget{variables-de-exceso-surplus-y-holgura-slack}{%
\subsection{variables de exceso (Surplus) y Holgura
(Slack)}\label{variables-de-exceso-surplus-y-holgura-slack}}

\hypertarget{aplicaciones-con-excel-y-lindo}{%
\subsection{aplicaciones con Excel y
lindo}\label{aplicaciones-con-excel-y-lindo}}

\hypertarget{soluciones-iniciales-muxe9todo-de-dos-fases-y-coeficiente-de-castigo}{%
\section{Soluciones iniciales Método de dos fases y Coeficiente de
castigo}\label{soluciones-iniciales-muxe9todo-de-dos-fases-y-coeficiente-de-castigo}}

\hypertarget{rompimiento-de-empates-en-el-muxe9todo-simplex}{%
\subsection{Rompimiento de empates en el método
simplex}\label{rompimiento-de-empates-en-el-muxe9todo-simplex}}

\hypertarget{otros-muxe9todos-de-soluciuxf3n}{%
\subsection{Otros métodos de
solución}\label{otros-muxe9todos-de-soluciuxf3n}}

\hypertarget{anuxe1lisis-de-la-soluciuxf3n-uxf3ptima}{%
\section{Análisis de la solución
óptima}\label{anuxe1lisis-de-la-soluciuxf3n-uxf3ptima}}

\hypertarget{definiciuxf3n-anuxe1lisis-de-sensibilidad}{%
\subsection{Definición análisis de
sensibilidad}\label{definiciuxf3n-anuxe1lisis-de-sensibilidad}}

\hypertarget{definiciuxf3n-del-modelo-dual}{%
\subsection{Definición del modelo
dual}\label{definiciuxf3n-del-modelo-dual}}

\hypertarget{procedimiento-de-conversiuxf3n-primal--dual}{%
\subsection{Procedimiento de Conversión Primal-
Dual}\label{procedimiento-de-conversiuxf3n-primal--dual}}

\hypertarget{relaciones-del-primal-dual}{%
\subsection{relaciones del
Primal-Dual}\label{relaciones-del-primal-dual}}

\hypertarget{interpretaciuxf3n-econuxf3mica-del-dual}{%
\subsection{Interpretación económica del
dual}\label{interpretaciuxf3n-econuxf3mica-del-dual}}

\hypertarget{cambios-en-los-coeficientes-de-la-funciuxf3n-objetivo.}{%
\subsection{Cambios en los coeficientes de la función
objetivo.}\label{cambios-en-los-coeficientes-de-la-funciuxf3n-objetivo.}}

\hypertarget{cambios-en-los-lados-derechos-de-las-restricciones}{%
\subsection{Cambios en los lados derechos de las
restricciones}\label{cambios-en-los-lados-derechos-de-las-restricciones}}

\hypertarget{ejercicios-de-aplicaciuxf3n-1}{%
\subsection{Ejercicios de
aplicación}\label{ejercicios-de-aplicaciuxf3n-1}}

\hypertarget{aplicaciones-con-excel-y-lindo-1}{%
\subsection{Aplicaciones con Excel y
lindo}\label{aplicaciones-con-excel-y-lindo-1}}

\hypertarget{optimizacion-lineal-entera}{%
\chapter{Optimizacion lineal entera}\label{optimizacion-lineal-entera}}

\hypertarget{optimizacion-no-lineal-continua}{%
\chapter{Optimizacion no lineal
continua}\label{optimizacion-no-lineal-continua}}

\hypertarget{problemas-de-inventario}{%
\chapter{Problemas de inventario}\label{problemas-de-inventario}}

\hypertarget{teoria-de-colas-o-fenomenos-de-espera}{%
\chapter{Teoria de colas o fenomenos de
espera}\label{teoria-de-colas-o-fenomenos-de-espera}}

\hypertarget{acknowledgments}{%
\section{Acknowledgments}\label{acknowledgments}}

I am grateful for the insightful comments offered by the anonymous peer
reviewers at Books \& Texts. The generosity and expertise of one and all
have improved this study in innumerable ways and saved me from many
errors; those that inevitably remain are entirely my own responsibility.

\bookmarksetup{startatroot}

\hypertarget{references}{%
\chapter*{References}\label{references}}
\addcontentsline{toc}{chapter}{References}

\markboth{References}{References}

\hypertarget{refs}{}
\begin{CSLReferences}{0}{0}
\end{CSLReferences}

\appendix
\addcontentsline{toc}{part}{Apéndices}


\printbibliography[title=apendice]


\end{document}
